
\section*{Introduction}


The high prevalence of osteoarthritis continues to
demand methods of diagnosing cartilage lesions, for which 
MRI is the most accurate non-invasive option.
There has been renewed interest
in SNR-efficient imaging sequences for
imaging cartilage, including various forms of
steady-state free precession (SSFP)
as well as driven equilibrium Fourier Transform (DEFT).  
We compare several of these sequences with existing
methods, both theoretically and in normal volunteers.  Results
show that the new steady-state methods have high potential for
knee imaging.


\section*{Theory}

Optimizing sequence parameters for SNR and contrast
requires consideration of the tissues of interest,
and the general dependence of SNR on sequence parameters.



\subsection*{SNR Relations}

The SNR of an imaging method is dependent on numerous factors
\cite{Macovski}.  
At a given field strength, the measured signal depends on the 
excitation sequence, T1, T2, resonant frequency, the coil
being used, and the voxel volume.  
The noise variance depends on the volume-sensitivity of the coil,
and the total acquisition time over which the noise is averaged.
It is commonly assumed that the SNR is inversely related to
readout bandwidth.  However, the total readout time to form the
image is a more important consideration.  Furthermore, it is important
to note that the SNR efficiency of 2D sequences is as high as that
of 3D sequences only if all slices are interleaved in one acquisition, and is
otherwise lower.  
For standard 2DFT/3DFT imaging, increasing the number of of averages
or increasing the FOV in either phase direction 
equivalently improve the SNR.  


\section*{Methods}

To fairly compare different cartilage imaging methods,
the parameters of each of six pulse sequences were optimized
to maximize cartilage SNR efficiency and CNR between cartilage
and synovial fluid.  We first describe each pulse sequence and
the relevent parameters for its optimization.  The optimized
sequences are then used to acquire {\em in vivo} images from 
which SNR and CNR numbers can be compared.


\subsection*{Pulse Sequences}

Two standard pulse sequences (SPGR and FSE) as well as 
four novel sequences (DEFT, FS-SSFP, LCSSFP and FEMR) are
described in the following sections.  Each sequence has 
specific considerations for its use in cartilage imaging.
Parameter optimizations were performed for all 
sequences to provide a fair basis for comparison.

\section*{Results}


For FSE, DEFT and SPGR, the SNR efficiency 
does not depend on resonant frequency.  
However, for FS-SSFP, LCSSFP and FEMR, the SNR-efficiency
varies considerably with resonant frequency.  
Figure \ref{fig:snrvsfreq} shows the relative SNR-efficiency
of the six sequences as a function of resonant frequency.
Note that the signal ``passband'' around 0~Hz is wider for
FS-SSFP than for LCSSFP or FEMR.  For all three steady-state
sequences, a $\pm 30$~Hz variation was assumed in calculating
the numerical SNR-efficiency.


\section*{Discussion}

The demand for non-invasive imaging of articular
cartilage is increasing.  
Although musculoskeletal MR imaging is already
common, it is important to continuously re-evaluate 
clinical techniques, especially as new MR sequences
are developed.  We have compared two standard cartilage
imaging protocols with four new, efficient imaging methods.
Results show that these new imaging methods all deserve
consideration for clinical use.  Before drawing conclusions,
we summarize the issues and results for each of the methods
here.


\section*{Conclusion}

Three new steady-state imaging sequences all simultaneously 
provide comparable or higher cartilage SNR efficiency 
as well as significantly higher cartilage-fluid contrast 
compared with FSE and SPGR sequences.  
Clinical comparisons of these sequences will be 
essential in evaluating the efficacy of these new sequences 
for diagnosis of other knee disorders such as ligament or
meniscal tears, though early results are very encouraging.



